%%=============================================================================
%% Inleiding
%%=============================================================================

\chapter{Inleiding}
\label{ch:inleiding}
Cisco netwerkapparatuur is een van de meest gebruikte in zijn soort binnen bedrijfsnetwerken. Grote netwerken die door verschillende bedrijven over heel de wereld aanwezig zijn worden vaak ondersteund met deze apparaten. Hoewel deze vaak rotsvast zijn en nog altijd security updates krijgen, kan een menselijk foutje er wel nog eens voor zorgen dat het even de mist ingaat.\autocite{ciscoLatestUpdate} Het instellen van deze apparaten is dan ook nog altijd een heel omslachtige taak en vereist heel wat manuele tussenkomsten. Hoe groter het bedrijf, hoe meer tijd hier dan ook inkruipt als er bijvoorbeeld een kleine wijziging op alle apparaten moet gebeuren. De laatste jaren is er heel wat vooruitgang geboekt bij het automatiseren van servers. Nu is dan ook de nood gekomen om hetzelfde te doen bij netwerkapparaten. 
\\

Het configureren van netwerkapparatuur kan dus heel wat tijd in beslag nemen. Hoewel er vaak bij nieuwere apparatuur mogelijkheden zijn om configuratie te automatiseren, zoals bijvoorbeeld met Cisco's eigen NX-OS die draait op zijn Nexus apparatuur, is het toch vaak vervelend dat er geen eenduidige manier is om dit te doen. \autocite{ciscoNxosPDF} Ieder merk heeft een eigen implementatie. Hiervoor zijn er de laatste enkele nieuwe mogelijkheden aan het opduiken. Waar er dus meer onderzoek naar gedaan zal worden.
\\

Vorig jaar heeft Dries Vandooren een gelijkaardig onderzoek uitgevoerd naar automatisering voor netwerkapparatuur met Ansible. Hieruit bleek dat bij hem Ansible nog niet klaar was om gebruikt te worden binnen een productieomgeving. In de tussentijd is er al heel wat extra vooruitgang geboekt en is het dan ook interessant om te kijken in hoeverre dit gebruikt kan worden voor in productie.\autocite{Vandooren2015}

Er zijn drie methodes die in het onderzoek vergeleken zullen worden. De eerste is hoe tegenwoordig vaak gewerkt wordt, de andere twee zijn beide geautomatiseerde technieken.
\begin{itemize}
\item Volledige Cisco configuratie's importeren 
\item Anible's officiële release van Network Tools (beschikbaar vanaf Ansible 2.1)
\item NAPALM: Network Automation and Programmability Abstraction Layer with Multivendor support.
\\
\end{itemize}

\section{Stand van zaken}
\label{sec:stand-van-zaken}
Cisco IOS is een NOS (of Network Operating System). Dit is een van de oudste besturingssystemen die nog dagelijks gebruikt wordt op netwerkapparatuur. Het is gekend om heel robuust te zijn en is dus vaak aan te raden wanneer er nood is aan een iets geavanceerdere netwerkopstelling indien de nodige kennis aanwezig is. IOS staat voor Internetworking Operating Systems en gaat al een hele periode mee, de eerste versies verschenen in 1990. Hoewel dit voor informaticastandaarden al een afgeschreven iets zou zijn krijgen apparaten die bijvoorbeeld in 2006 geproduceerd zijn nog altijd security updates en zijn ze voor een bedrijfsnetwerk dan ook nog uiterst geschikt. \autocite{historyOfCiscoCli} 
\\

Het enige waarbij deze apparatuur wel duidelijk toont dat deze toch wel wat verouderd begint te worden is dus automatisatie en  provisioning. Bij nieuwere apparatuur worden er vaak ook al verschillende besturingssystemen meegeleverd die vaak heel wat meer automatisatie mogelijkheden ingebouwd hebben. Een voorbeeld hiervaan is het laatste nieuwe besturingssysteem van Cisco genaamd NX-OS die op de productlijn met naam 'Nexus' draait. Deze wordt door hen beschreven als 'Open, Progammable,  Agile'. Dit lost natuurlijk nog altijd het probleem niet op dat we hebben met onze oudere maar wel nog functionele apparatuur. \autocite{ciscoNxos}
\\

De huidige uitwerking om een apparaat te configureren bestaat vaak uit twee methodes. De eerste is het handmatig uitvoeren van ieder commando. De tweede is een copy te nemen van de running config of startup config en deze te plakken in het terminal venster van een gelijk apparaat. Dit kan je ook doen door de config op te slaan op een TFTP server en deze dan van daaruit laten ophalen door een nog te configureren apparaat. Deze spreekt dan de TFTP server aan en neemt de configuratie op.

\subsection{Wat is Ansible}
\label{sec:whatisansible}

Dit is waar Ansible een noodzakelijk iets is. Ansible is iets die de laatste jaren een heel snelle opmars aan het maken is. Maar wat is het nu precies? Ansible is een project die het automatiseren van servers over ssh mogelijk maakt. Dit klinkt niet heel spectaculair maar dat is ook niet de hoofdreden waarvoor je als netwerkbeheerder er gebruik van moet maken. Ansible maakt het mogelijk om op een heel gestructureerde manier te bepalen wat op een server gedaan wordt, geïnstalleerd wordt, naar gekopieerd wordt etc. Het belangrijkste aspect moet dan toch wel zijn dat alles in een heel leesbare en eenvoudig verstaanbare vorm opgeslagen wordt in documentjes die in yaml geschreven zijn. Indien je niet veel verstaat van Ansible is het toch nog mogelijk om een playbook te bekijken en te zien wat er precies zal gebeuren als deze uitgevoerd wordt.
\\

Nu zullen we even dieper ingaan van waar Ansible nu precies uit ontstaan is. De geschiedenis van Ansible zoals beschreven door oprichter \textcite{historyAnsible}.
\\

Ansible is dus nog iets vrij recent, het is iets dat sinds 2012 in ontwikkeling is, startende bij een Red Hat Emerging Technologies groep in 2006. Hier werkten werknemers vrij aan wat zelf interessant leek voor de klant. Hieruit kwam 'Cobbler' voort, dit had als doel het beheren van datacenters eenvoudiger te maken.  Dit werd vrij snel groot en kreeg mogelijkheden om bijvoorbeeld DHCP en DNS te beheren in combinatie met dit pakket. Dit werd zo groot dat het gebruikt werd om de grootste DNS netwerken te beheren, heel wat grotere gameservers zoals Modern Warfare van Call of Duty en dan nog enkele kritische zaken zoals financiële onderdelen op Wall Street. 
\\

Toen kwamen configuratie managementtools zoals Puppet en Chef op de proppen. Deze waren in combinatie met Cobbler een sterke combinatie. Cobbler stelde het basissysteem in en dan werd Puppet gebruikt om de virtuele machines op het apparaat te configureren. Hiermee bleken er nog wel wat problemen te zijn waardoor deze oplossing voor verschillende situaties niet ideaal was. Met Puppet was het toen bijvoorbeeld niet mogelijk om een server te herstarten.
\\

Om dit probleem op te lossen werd Func in het leven geroepen. Het concept was vrij eenvoudig: een centrale server die andere servers aanspreekt om bepaalde taken uit te voeren. Dit is vrij gelijkaardig aan een 'Master/slave' configuratie waarbij de 'slaves' luisteren naar de 'master'. Hoewel het niet een heel groot succes werd is het toch gebruikt geweest door Tumblr en zijn er 2 spin-offs uit voortgekomen die elk ook in hun resultaten niet volledig waren. Dit duidt er wel op dat er in de correcte richting gewerkt werd.
\\

Deze periode is het moment dat de populariteit van de term 'DevOps' heel erg toenam. Dit werd toen nog vooral geassocieerd met Cobbler, Puppet en Func. Daarna kreeg de betekenis een bredere zin die meer duidde op de cultuur rond het automatiseren van infrastructuur. 
Zelf is de oprichter van Ansible nog een lange tijd na deze periode een Puppet voorstander geweest maar het bleek al snel dat er een eenvoudigere taal nodig was om een groter publiek te kunnen bereiken. Zo wou hij ook de enthousiasteling bereiken naast de server expert.
\\

Door alles samen te zetten kwam de nood voor Ansible. Zoals de oprichter \textcite{historyAnsible}  het zelf zegt ' it was mostly to build the tool that you could not use for six months, come back to, and still remember'. Dus rond februari 2012 werd de fundatie gelegd voor het volledige project, omdat de ontwikkelaar al heel wat kennissen had binnen de DevOps wereld sloeg het vrij snel aan. Samen me de community werd en wordt alles onderhouden en aangevuld. \autocite{historyAnsible}
\\

Nu een viertal jaar later is Ansible uitgegroeid tot een vaste waarde voor het automatiseren van infrastructuur. In de tussentijd is er nog de toevoeging van onder andere Ansible Tower (de betaalde versie van Ansible voor grotere bedrijven). Dit biedt de mogelijkheid om een overzicht te houden van je hele serverinfrastructuur en maakt het bijhouden van Ansible playbooks eenvoudig en overzichtelijk. Dit wordt vaak aangekaart op de website, maar hiernaast blijft het gewone Ansible ook bestaan voor de community en enthousiastelingen. \autocite{ansibleTowerRelease}
\\

Dan sinds kort de laatste belangrijke toevoeging, vooral ook voor dit onderzoek. De toevoeging van de network-modules aan Ansible. Vanaf versie 2.0 is de eerste mogelijkheid beschikbaar voor het automatiseren, maar het is pas vanaf versie 2.1 die op 25 mei 2016 uitgebracht werd dat deze ook in de core-modules zijn ingesloten. Deze bevat de mogelijkheid om veel netwerkapparatuur van verschillende merken aan te spreken via een Ansible playbook. Enkele voorbeelden hiervan zijn Junos VyOS en nuttig voor ons onderzoek: IOS.
 \autocite{ansible2}\autocite{ansiblechangelog}
 
\subsection{Wat is NAPALM}
\label{sec:napalm}
NAPALM staat voor 'Network Automation and Programmability Abstraction Layer with Multivendor support'. In andere woorden iets met een gelijkaardig doel als wat Ansible probeert te bereiken met hun network modules, het configureren van netwerkapparatuur vereenvoudigen en hier meer bepaald deze van verschillende merken. Net als ansible is alles ook open source terug te vinden op een Github pagina waardoor ook hier community kan aan meewerken. Het nadeel aan deze optie is dat deze minder aanhangers zal hebben omdat het iets volledig nieuw op zich is speciaal voor netwerkapparatuur.\autocite{napalmGithub} 
\\

De eerste commits op Github dateren van begin 2015 waardoor het nog niet zo lang bestaat op zich maar qua netwerkautomatisatie op zich dus wel al een stuk langer in ontwikkeling dan de modules van Ansible. Deze wordt nog altijd actief ondersteund waardoor het wel een goede oplossing kan zijn voor een huidig netwerk te automatiseren. Support voor verschillende apparaten is goed en duidelijk gedocumenteerd met welk netwerk besturingssysteem wat precies mogelijk is. Hieruit is vrij snel duidelijk dat IOS het grootste deel van de mogelijkheden ondersteund.\autocite{napalmSupport}

\subsection{Overige mogelijkheden}
\label{sec:overige}
Als laatste bestaan er nog enkele kleinere opties om het opzetten van een netwerk met Cisco IOS apparatuur te vereenvoudigen.  Dit is dan niet de automatisatie zoals gewoonlijk maar het is wel eenvoudiger dan het gewoon configureren van een apparaat. Omdat deze een manier bieden die vandaag nog vaak gebruikt worden zijn deze dus ook meegenomen in het onderzoek.
\\

Een van de oudere methodes om het opzetten van een switch eenvoudiger te maken is het kopiëren van de huidige configuratie van een netwerk apparaat naar een gelijke die gelijkaardig geconfigureerd moet zijn. Op die manier moet je toch al heel wat minder werk doen. Het nadeel hiermee is dat je welk nog elk apparaat afzonderlijk moet configureren al is het dan wel niet de gehele configuratie. Het voordeel van deze methode (tegenover het gewoon handmatig configureren) kan zijn dat als je bijvoorbeeld 10 switches hebt die elk dezelfde configuratie moeten hebben. Het enige verschil zal dan de ip adressering zijn wat dan uiteindelijk een kleinere wijziging is en een stuk minder werk is dan handmatig alle commando's opnieuw invoeren. Een ander nadeel kan zijn dat dit nog vrij handmatig is waardoor een foutje toch nog af en toe kan voorkomen bij het te snel willen zijn en je dus iets over het hoofd zou kunnen zien.
\\

Een laatste die nog vermeld moet worden is de 'ansible-cisco-snmp' module van Networklore. Deze is vorig jaar uitgebreid besproken geweest in het onderzoek van Dries Vandooren en wordt daarom hier niet verder besproken. Voor dieper onderzoek kan ik verwijzen naar zijn werk. Bij de conclusie zal er ook rekening gehouden worden met zijn bevindingen. Een interessant iets wat wel op te merken valt aan de module is dat ongeveer sinds de network modules van Ansible zelf uitgekomen zijn er geen commits meer toegevoegd zijn aan de repository op Github.\autocite{Vandooren2015}


\section{Probleemstelling en Onderzoeksvragen}
\label{sec:onderzoeksvragen}
Tegenwoordig wordt nog heel veel netwerkbeheer gedaan aan de hand van manueel wijzigingen doorvoeren en configuraties uitschrijven. Dit kan heel tijdrovend zijn en zorgt dus voor veel overhead. Tijd is iets die veel IT'ers te kort komen en door het huidig werk te automatiseren zou er dus weer meer tijd vrijkomen voor andere taken.
\\

Een voorbeeld uit het echte leven zou het volgende kunnen zijn.
In België heb je een van de grootste Lanparties genaamd 'Frag-O-Matic'. Deze hebben jaarlijks 2 edities waar ongeveer een 1000 tal gamers op afkomen, het opzetten van dergelijke structuur is dan ook een heel werk. In het netwerkteam alleen al zitten een 15 tal enthousiastelingen die naast hun vaak full time job of school nog heel wat avonden tijd vrij maken om het netwerk voor de volgende editie op punt te stellen. Alles wordt uitgetekend, uitgewerkt en uiteindelijk getest. Zelf al hebben zij hun grootste deel van de configuratie op voorhand klaar zou het handiger zijn moest er een automatisch manier zijn om alles in 1 keer te provisionen en zo het werk van dit team te verlichten. 
\\

Op die manier zou het per editie in plaats van alles opnieuw te kopiëren zijn eerder de playbooks eens doorlopen en aanpassen of aanvullen waar nodig. Een testopstelling kan dan ook op een korte tijd opgezet worden om de werking van bepaalde onderdelen van het netwerk te testen.
\\

Dit is een van de vele mogelijkheden die mogelijk gemaakt worden via het automatiseren van Cisco switches en routers. Er zullen naast dit team nog veel andere teams en netwerkbeheerders zijn die een geautomatiseerde omgeving op prijs zouden stellen.

Mijn onderzoeksvragen sluiten dan ook vooral aan op de mogelijkheid van het gebruik van Ansible voor het automatiseren binnen de scope van netwerkbeheer.
\begin{itemize}
\item Is het mogelijk om met Ansible een netwerk te provisionen op Cisco IOS apparatuur.
\item In hoeverre is de implementatie volwassen om dit voor het dagelijks leven te gebruiken en dit te laten integreren bij bedrijven.
\item Zijn er dingen die niet mogelijk zijn met Ansible.
\item Wat is de minimumvereiste voor de apparatuur om aan de slag te kunnen gaan en playbooks te pushen vanaf Ansible.
\\
\end{itemize}


%% TODO:
%% Uit je probleemstelling moet duidelijk zijn dat je onderzoek een meerwaarde
%% heeft voor een concrete doelgroep (bv. een bedrijf).
%%
%% Wees zo concreet mogelijk bij het formuleren van je
%% onderzoeksvra(a)g(en). Een onderzoeksvraag is trouwens iets waar nog
%% niemand op dit moment een antwoord heeft (voor zover je kan nagaan).

\section{Opzet van deze bachelorproef}
\label{sec:opzet-bachelorproef}

%% TODO: Het is gebruikelijk aan het einde van de inleiding een overzicht te
%% geven van de opbouw van de rest van de tekst. Deze sectie bevat al een aanzet
%% die je kan aanvullen/aanpassen in functie van je eigen tekst.

De rest van deze bachelorproef is als volgt opgebouwd:

In Hoofdstuk~\ref{ch:methodologie} wordt de methodologie toegelicht en worden de gebruikte onderzoekstechnieken besproken om een antwoord te kunnen formuleren op de onderzoeksvragen.

%% TODO: Vul hier aan voor je eigen hoofstukken, één of twee zinnen per hoofdstuk
Hoofdstuk~\ref{ch:testopstelling} is waar we een kijkje nemen naar hoe de testopstelling er precies uitziet en wat we precies nodig hebben om alles werkende te krijgen.

In Hoofdstuk~\ref{ch:configuratie} wordt de basisinstelling van het Cisco apparaat vastgelegd en wat het te bereiken resultaat moet zijn. 

In Hoofdstuk~\ref{ch:ansible} gaan we verder in op de bereiken van de doelstelling met het Ansible framework. 

In Hoofdstuk~\ref{ch:napalm} kijken we naar de mogelijkheden die beschikbaar zijn met het nog vrij onbekende NAPALM. 

In Hoofdstuk~\ref{ch:conclusie}, tenslotte, wordt de conclusie gegeven en een antwoord geformuleerd op de onderzoeksvragen. Daarbij wordt ook een aanzet gegeven voor toekomstig onderzoek binnen dit domein.

