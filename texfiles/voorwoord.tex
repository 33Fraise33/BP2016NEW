%%=============================================================================
%% Voorwoord
%%=============================================================================

\chapter*{Voorwoord}
\label{ch:voorwoord}

%% TODO:
%% Het voorwoord is het enige deel van de bachelorproef waar je vanuit je
%% eigen standpunt (``ik-vorm'') mag schrijven. Je kan hier bv. motiveren
%% waarom jij het onderwerp wil bespreken.
%% Vergeet ook niet te bedanken wie je geholpen/gesteund/... heeft

Het onderwerp in kwestie is gekozen geweest door mijn interesse in Ansible op zich. Twee jaar geleden heb ik de eerste hands-on gekregen in de les van Mr. Van Vreckem waar het me al snel aansprak. Nu enkele jaren later blijkt Ansible op veel verschillende plaatsen al gebruikt te worden, dit merk ik ook in het dagelijkse leven. 
\\

Een van de aspecten waar ik hier mee in aanraking kom is binnen 'Frag-O-Matic' dit is een Lanparty in België die gemiddeld een duizendtal bezoekers trekt per editie. Hier merkte ik dat zo goed als alles die zij deden op servers in playbooks gegoten waren voor Ansible. Een van de aspecten waar wel nog veel werk aan is zijn de netwerkswitches en andere dergelijke apparatuur. 
\\
Hier begon bij mij een belletje rinkelen over het onderwerp: Ansible met netwerkapparatuur. Na een beetje zoeken merkte ik snel dat dit het geschikte moment was om het onderzoek te verrichten omdat de technologie nog maar sinds kort als eerste stable versie naar buiten werd gebracht. Al snel werd mij terug duidelijk wat er nu weer zo fantastisch is aan Ansible.
\\

Als laatste wil ik Stefanie Geldof en haar broer bedanken voor het gebruik van de Cisco 2960 Series switch, zonder dit was het niet mogelijk geweest. 
Ook BeSports en Frag-O-Matic wil ik bedanken om mij in aanraking te brengen met het idee voor iets dergelijk. Automatisatie voor een Lanparty lijkt me heel handig maar het is door hen dat ik met het idee kwam.
\\

Ik wens de lezer veel leesplezier toe, welkom in de wereld van automatisatie.