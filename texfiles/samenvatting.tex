%%=============================================================================
%% Samenvatting
%%=============================================================================

%% TODO: De "abstract" of samenvatting is een kernachtige (~ 1 blz. voor een
%% thesis) synthese van het document.
%%
%% Deze aspecten moeten zeker aan bod komen:
%% - Context: waarom is dit werk belangrijk?
%% - Nood: waarom moest dit onderzocht worden?
%% - Taak: wat heb je precies gedaan?
%% - Object: wat staat in dit document geschreven?
%% - Resultaat: wat was het resultaat?
%% - Conclusie: wat is/zijn de belangrijkste conclusie(s)?
%% - Perspectief: blijven er nog vragen open die in de toekomst nog kunnen
%%    onderzocht worden? Wat is een mogelijk vervolg voor jouw onderzoek?
%%
%% LET OP! Een samenvatting is GEEN voorwoord!

%%---------- Samenvatting -----------------------------------------------------
%%
%% De samenvatting in de hoofdtaal van het document

\chapter*{\IfLanguageName{dutch}{Samenvatting}{Abstract}}

In deze bachelorproef werd onderzoek verricht naar het automatiseren van netwerkapparatuur in combinatie met Ansible en een tegenhanger NAPALM. Dit voor het automatiseren van Cisco IOS apparatuur. 
\\

Eerst en vooral werd er gekeken naar de huidige stand van zaken bij de Cisco apparatuur. Uit onderzoek blijkt dat nog heel wat IOS apparaten gebruikt worden in de bedrijfswereld maar dat deze qua configuratie nog vrij hard achter lopen. Met de nieuwere apparatuur die NX-OS draait zijn er meer mogelijkheden hiervoor. Het nadeel is dan wel dat je vaak met een automatisatietechniek zit die niet echt standaard is en het dus extra moeilijk maakt wil je veel apparaten beheren.
\\

Als eerste grote automatisatietechniek werd gekeken naar de implementatie van Ansible. Dit is een speler die in de wereld van serverprovisioning en -automatisatie al heel wat volgers heeft. Maar nu werd er ook gekeken hoe zij het doen op vlak van netwerkautomatisatie. Na het onderzoek blijkt dat de laatste tijd heel wat vooruitgang is geboekt, bij het publiceren van dit werk zijn de network modules die hier gebruikt zijn pas enkele maanden uitgebracht. 
\\
Er werd een kleine opstelling gemaakt met een netwerkswitch van Cisco die IOS als besturingssysteem draaiende had en deze werd aangesloten op een gewone router. Hierna werd een configuratie aangemaakt uit een voorbeeld van een CCNA oefening. Deze configuratie werd daarna nagemaakt in Ansible en daarna werd dit ook geprobeerd met NAPALM.
\\

Uit de resultaten blijkt dat het opzetten van deze methodes nog niet zo eenvoudig is als het zou blijken, maar het ziet er wel hoopvol uit. Bij Ansible was het mogelijk om het apparaat in te stellen naar de configuratie die we wensten. Dit in combinatie met variabelen en templates zoals het gewoonlijk ook werkt bij Ansible. Jammer genoeg is dit nog niet op de volledig juiste manier omdat er momenteel nog een probleem is met de module 'ios\_config' die aangeraden wordt om configuraties te wijzigen op het IOS besturingssysteem. Er is gebruik gemaakt van 'ios\_command' voor dit onderzoek. Dit werkt ook correct maar is niet optimaal omdat deze niet kijkt naar de huidige configuratie.
\\

Hierna werd NAPALM bekijken, dit is een python framework die geschreven is specifiek voor het aansturen van netwerkapparaten waaronder dus ook Cisco IOS apparaten. De setup bleek iets moeilijker te zijn dan bij ansible maar eenmaal in de python shell leek alles vrij logisch. Jammer genoeg zijn er met IOS enkele problemen die er voor zorgden dat het verder werken met deze methode dus geen optie was binnen dit onderzoek. Ontwikkeling is wel nog actief dus hopelijk komt er hier snel een oplossing voor. Verder is deze methode iets anders dan de pure Ansible methode, deze stuurt een configuratie door en geeft de verschillen met de huidige configuratie terug. 
\\
Daarnaast is er nog een Ansible module die inwerkt op NAPALM. Dit zou nuttig kunnen zijn voor het testen na het doorvoeren van configuratie wijzigingen met de Ansible network modules. Jammer genoeg werkte NAPALM op zich al niet en is dit dus niet op de proef gesteld.
\\

Als besluit werd er gesteld dat Ansible voor dagelijks gebruik al kan gebruikt worden omdat dit op het moment van schrijven de beste methode voor handen is. De release van Ansible 2.2 zou ook niet heel ver af meer moeten zijn, deze brengt dan nog verbeteringen mee voor deze modules waardoor de mogelijkheden er enkel maar beter op zullen worden.